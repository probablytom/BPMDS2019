\documentclass[draft,12pt]{article}

% Article metadata
\title{Generating Plausibly Real Synthetic Data using Dynamic Fuzzing}
\author{Tom Wallis and Tim Storer}
\institute{Glasgow University}
\date{}

% general packages
\usepackage{indentfirst}
\usepackage[textsize=tiny]{todonotes}
\usepackage{cleveref}
\usepackage{float}

% Packages for code highlighting
\usepackage{color}
\usepackage{listings}
\usepackage{caption}

% bibliography stuff
\usepackage{natbib}
\bibliographystyle{abbrvnat}
\setcitestyle{authoryear,open={(},close={)}}

% Pseudocode environment nicked from 
% https://tex.stackexchange.com/questions/111116/what-is-the-best-looking-pseudo-code-package


\lstnewenvironment{algorithm}[1][] %defines the algorithm listing environment
{   
    \captionsetup{labelsep=colon} %defines the caption setup for: it ises label format as the declared caption label above and makes label and caption text to be separated by a ':'
    \lstset{ %this is the stype
        mathescape=true,
        frame=tB,
        numbers=left, 
        numberstyle=\tiny,
        basicstyle=\scriptsize, 
        keywordstyle=\color{black}\bfseries\em,
        keywords={,input, output, return, datatype, function, in, if, else, foreach, while, begin, end, } %add the keywords you want, or load a language as Rubens explains in his comment above.
        numbers=left,
        xleftmargin=.04\textwidth,
        #1 % this is to add specific settings to an usage of this environment (for instnce, the caption and referable label)
    }
}
{}

\begin{document}
\maketitle

\begin{abstract}
Science requires data. Some work exists toward producing synthetic data sets
for testing information systems in a controlled way, but never addresses
whether the data sets are representative of the real world. To produce data
sets that can plausibly be described as ``realistic'', we note that it is
necessary to account for unexpected behaviours which produce noise in
empirically sourced data sets. We present a novel approach to simulation,
where variance in behaviour is captured as a component of a model, and used to
introduce errors in the behaviour represented by such a model. These possibly
erroneous behaviours are simulated, introducing realistic noise to generated
data sets. To achieve this, a new paradigm for the construction of business
process models is introduced. Implementation of models under this new paradigm
is detailed in full. The paradigm lends a new perspective on modelling with
impacts on the broader BPMDS community, and this future work is discussed.
\end{abstract}




\section{Designing with Synthetic Data}
% There's some techniques at the moment for designing with synthetic data: secsy
% and the sap paper. They give us a kind of feedback loop, in that they give us
% a way to generate data with the intention of designing a scientific experiment
% or an industrial tool which can be tested. So, we get our standard feedback
% loop by inferring from their process.

% More than that, we can infer from their work that the feedback loop we
% proposed in the meeting makes sense --- they motivate that feedback loop by
% either asserting it exists and improving it with their work, or making it
% possible. Maybe?

% Once we've got *their* feedback loop, we can assert that the loop has a flaw:
% it doesn't deal with emergent properties in an especially good way. We can do
% better. 
Controlled, representative data is important in a variety of areas, for a
variety of reasons. Scientists require controlled data to run experiments with.
Industry requires data for the improvement of products, services and internal
processes.
\par

Data for business processes with these properties can be difficult to obtain.
As in any complex system, data generated from one business process is often
incomparable with another, because small emergent phenomena can exert massive
effects on the output of the 
\par


\section{Better Feedback Loops}
The knock-on effects of variance in empirical processes presents an issue for
the existing paradigm, and simulating these knock-on effects is key to using
these models for realistic synthesis. This critical flaw with the design of
these feedback loops can be fixed by a focus on improving the way in which the
models used in the loops reflect the real world.
\par



\section{Generating Plausibly Real Synthetic Data}



\section{Further notes on Dynamic Fuzzing}




\section{Existing Synthetic Data with Variance Techniques}

% SecSY, SAP paper


