\begin{abstract}
Science requires data. Some work exists toward producing synthetic data sets
for testing information systems in a controlled way, but never addresses
whether the data sets are representative of the real world. To produce data
sets that can plausibly be described as ``realistic'', we note that it is
necessary to account for unexpected behaviours which produce noise in
empirically sourced data sets. We present a novel approach to simulation,
where variance in behaviour is captured as a component of a model, and used to
introduce errors in the behaviour represented by such a model. These possibly
erroneous behaviours are simulated, introducing realistic noise to generated
data sets. To achieve this, a new paradigm for the construction of business
process models is introduced. Implementation of models under this new paradigm
is detailed in full. The paradigm lends a new perspective on modelling with
impacts on the broader BPMDS community, and this future work is discussed.
\end{abstract}
