% Article metadata
\title{Generating Plausibly Real Synthetic Data using Dynamic Fuzzing}
\author{Tom Wallis and Tim Storer}
\institute{Glasgow University}
\date{}

% general packages
\usepackage{indentfirst}
\usepackage[textsize=tiny]{todonotes}
\usepackage{cleveref}
\usepackage{float}

% Packages for code highlighting
\usepackage{color}
\usepackage{listings}
\usepackage{caption}

% bibliography stuff
\usepackage{natbib}
\bibliographystyle{abbrvnat}
\setcitestyle{authoryear,open={(},close={)}}

% Pseudocode environment nicked from 
% https://tex.stackexchange.com/questions/111116/what-is-the-best-looking-pseudo-code-package


\lstnewenvironment{algorithm}[1][] %defines the algorithm listing environment
{   
    \captionsetup{labelsep=colon} %defines the caption setup for: it ises label format as the declared caption label above and makes label and caption text to be separated by a ':'
    \lstset{ %this is the stype
        mathescape=true,
        frame=tB,
        numbers=left, 
        numberstyle=\tiny,
        basicstyle=\scriptsize, 
        keywordstyle=\color{black}\bfseries\em,
        keywords={,input, output, return, datatype, function, in, if, else, foreach, while, begin, end, } %add the keywords you want, or load a language as Rubens explains in his comment above.
        numbers=left,
        xleftmargin=.04\textwidth,
        #1 % this is to add specific settings to an usage of this environment (for instnce, the caption and referable label)
    }
}
{}