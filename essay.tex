\documentclass[12pt,draft]{article}

% Article metadata
\title{Generating Plausibly Real Synthetic Data using Dynamic Fuzzing}
\author{Tom Wallis and Tim Storer}
\institute{Glasgow University}
\date{}

% general packages
\usepackage{indentfirst}
\usepackage[textsize=tiny]{todonotes}
\usepackage{cleveref}
\usepackage{float}

% Packages for code highlighting
\usepackage{color}
\usepackage{listings}
\usepackage{caption}

% bibliography stuff
\usepackage{natbib}
\bibliographystyle{abbrvnat}
\setcitestyle{authoryear,open={(},close={)}}

% Pseudocode environment nicked from 
% https://tex.stackexchange.com/questions/111116/what-is-the-best-looking-pseudo-code-package


\lstnewenvironment{algorithm}[1][] %defines the algorithm listing environment
{   
    \captionsetup{labelsep=colon} %defines the caption setup for: it ises label format as the declared caption label above and makes label and caption text to be separated by a ':'
    \lstset{ %this is the stype
        mathescape=true,
        frame=tB,
        numbers=left, 
        numberstyle=\tiny,
        basicstyle=\scriptsize, 
        keywordstyle=\color{black}\bfseries\em,
        keywords={,input, output, return, datatype, function, in, if, else, foreach, while, begin, end, } %add the keywords you want, or load a language as Rubens explains in his comment above.
        numbers=left,
        xleftmargin=.04\textwidth,
        #1 % this is to add specific settings to an usage of this environment (for instnce, the caption and referable label)
    }
}
{}

\begin{document}
\maketitle
\begin{abstract}
Science requires data. Some work exists toward producing synthetic data sets
for testing information systems in a controlled way, but never addresses
whether the data sets are representative of the real world. To produce data
sets that can plausibly be described as ``realistic'', we note that it is
necessary to account for unexpected behaviours which produce noise in
empirically sourced data sets. We present a novel approach to simulation,
where variance in behaviour is captured as a component of a model, and used to
introduce errors in the behaviour represented by such a model. These possibly
erroneous behaviours are simulated, introducing realistic noise to generated
data sets. To achieve this, a new paradigm for the construction of business
process models is introduced. Implementation of models under this new paradigm
is detailed in full. The paradigm lends a new perspective on modelling with
impacts on the broader BPMDS community, and this future work is discussed.
\end{abstract}


\section{Introduction}
The verification of scientific work is a process which relies on the
availability of test data which is controlled and abundant.
\par

For many fields, and particularly for traditional scientific process, such data
is readily available. Simple physical data points, such as the velocity or mass
of the subject of study, are easily measured, in turn providing ample
high-quality data sets. As subjects of study become more complex, however, data
collection becomes an increasingly complex task. Business process modelling is
one example of the study of a complex, difficult-to-measure system.
\par

Business process modelling in particular has a unique set of challenges
available in collecting data: high-quality data sets are especially difficult to
produce, and those who do produce them are typically the businesses who manage
the process subject to analysis. This means that high-quality collected data
sets tend to be silo-ed into industry, and are not publicly available.
\par

The affect of this situation is to handicap available scientific process in a
variety of fields, notably business process modelling. Verification of tooling
is difficult, and evaluation rarely makes use of available test data, because
such data cannot feasibly be attained in an academic research context without
very close ties to industry. Even with industrial ties, this situation poses an
issue, as the scientific process depends on the notion of reproduction.
Without available test data, the reproduction of any evaluation which
\emph{does} make use of test data is tricky at best and intractable to
impossible at worst.
\par

The conceit of this work is that if real-world test data cannot be found,
perhaps it can be sufficiently well faked. This paper introduces an alternative
approach to business process modelling using \emph{dynamic fuzzing}, an
experimental software engineering paradigm. We show that this approach
substantiates an alternative paradigm for business process modelling, and
discuss the ways by which a dynamic fuzzing approach might augment current
efforts in the BPMDS community to synthesise event log data which can plausibly
be described as ``realistic''. Realism is introduced to the model by simulating
flaws in the real-world execution of a business process, contingent on
properties of both the actor executing the business process and the
environemntal concerns which might alter that actor's behaviour in ``noisy''
ways.
\par

We assert that movement towards particularly realistic modelling makes possible
the generation of controlled data sets which exhibit a variety of traits
expected in the real world: noise, security violation, partial or missing log
data, and other variation from the expected behaviour that one might anticipate
in an empirical data set.
\par

\todo{I think it might make more sense to explore DF, then apply it to business
  process modelling and explore it within the context of ``transformative
  BPMDS'', and compare to existing methods and show that application of DF in
  this field actually addresses a number of issues in the state of the art.
  Possibly refactor the structure.}

In \cref{sec:curr-methods}, existing approaches to data synthesis are discussed;
the methods are assessed and their flaws noted. Dynamic fuzzing is then examined
as an approach in \cref{sec:dynamic-fuzzing}, and the paradigm introduced and
fully described. Then, in \cref{sec:implementations-and-modelling}, example
methods for building processes are discussed for the reader to implement
business process models which make use of dynamic fuzzing themselves, and
conversion from a BPMN-style format to include this new paradigm is discussed.
An evaluation of the philosophy with regards realistic log synthesis is
presented in \cref{sec:evaluation}, with an examination of the broader
implications for the BPMDS community in \cref{sec:implications}.



\section{Current Methods}\label{sec:curr-methods}
% Methods built on process structure trees


\subsection{SecSY}
% From the SecSY paper
SecSY\todo{CITE} is a tool produced with the intention of solving a lack of
``mechanisms for business process security monitoring and auditing''. In
particular, the work addresses the need for controlled log generation for
testing purposes with regards flexible and varying processes which exhibit
non-compliance. The authors' rationale is that, as the information systems
community increasingly develop tools which can test for certain traits in an
event log --- such as security policy violation --- their testing requires
suitable test data. A set of logs which exhibit the sought property, and a set
which do not, should be producable from a specification of ideal behaviour.
\par

The tool produces event logs exhibiting arbitrary properties which one might
want to observe in a controlled manner. It does this by executing a provided
process specification according to a ``context'', and recording the associated
event log (the simulation's ``trace''). The trace is then edited via a set of
transformers. The transformers used are well documented in \todo{CITE}, but
roughly correspond to switching gateway types, swapping order of task execution,
editing associated actors to introduce violation of authenatication, and editing
associated actors so that the same actor did not execute all of a set of
actions.
\par

The authors note that SecSY is ``capable of producing log files of industrial
complexity'', and that these logs are readily available to be used in testing by
the process mining community. The tool's evaluation affirms this by noting the
tool's runtime and that the transformers they detail are ``effective''. This
does not guarantee that transformers produce realistic data, however. We can
surmise that a transformer-based approach can be reliably and efficiently
implemented, but have no guarantees as to the \emph{realism} of the log it might
produce.
\par

We can expect that such a method would produce a realistic model only with
significant effort and changes. In the real world, logs produced with certain
pieces of variance ought to exhibit the impact of that variance on the decisions
made by agents in later stages of a process. Put another way, variance in the
real world produces state changes which have an effect on future decisions
actors might make. Post-processing on event logs would need to take this state
into account to predict the ``knock-on effects'' of behavioural variance, and
each sequential effect the initial change would make on state may have
cumulative impacts on the overall workflow execution.
\par

Therefore, while this method shows initial promise, amending it to satisfy the
requirement of realism in a generated event log seems intractable.
\par

\subsection{Generating Business Process Variants}  % What's the name of this actual technique? TODO: Re-read the paper...
Similar work has been undertaken by researchers at SAP and Ryerson University,
Canada. Researchers 

\subsection{Limitations of these Approaches}
\label{subsec:existing_work_limitations}

The limitation of this
approach is evident when we consider what the impact of contingent behaviour is
on certain details of the world they are intended to simulate. In the real
world, the changes applied in ordinary process structure tree models fail to
represent how changes at one stage in a procedure impact future behaviour.\par

A toy example of the real world might be a worker following a simple workflow on
a factory floor. The worker's workflow can be represented through a plethora of
standard modelling techniques, and the process fits the block-structured
workflow modelling required for analysis as a Process Structure Tree. A piece of
contingent behaviour which could be represented via these models might be the
actor executing the workflow being distracted.\par

Existing techniques, such as \todo{CITE}, represent these changes as an
amendment to the event log produced by a simulation from a Process Structure
Tree. \par




\section{Dynamic Fuzzing}\label{sec:dynamic-fuzzing}

Program fuzzing is an experimental software engineering paradigm where programs
are rewritten during their execution. When program fuzzing is employed, a
program segment\footnote{Typically a function or method.} is annotated with a
second function, called a ``fuzzer''. When the original program segment is
invoked, a high-level representation of the process it
represents\footnote{Typically a graph representation, such as a segment of a
  Finite State Automata or an Abstract Syntax Tree.} is delivered
as an argument to the fuzzer, which returns an altered version of it to be run
in lieu of the original. The interaction between original source and fuzzer can
have a variety of nuances, which can influence the impact of dynamic fuzzing as
a technique.

\begin{description}
  \item[Fuzzer] Fuzzer definition
  \item[]
  
\end{description}


\section{Implementations and Modelling}
\label{sec:implementations-and-modelling}



\section{Realistic Log Synthesis}\label{sec:evaluation}



\section{Implications}\label{sec:implications}




\section{Conclusion}\label{sec:conclusion}


\end{document}
