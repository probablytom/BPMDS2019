\documentclass[12pt,draft]{article}

% Article metadata
\title{Generating Plausibly Real Synthetic Data using Dynamic Fuzzing}
\author{Tom Wallis and Tim Storer}
\institute{Glasgow University}
\date{}

% general packages
\usepackage{indentfirst}
\usepackage[textsize=tiny]{todonotes}
\usepackage{cleveref}
\usepackage{float}

% Packages for code highlighting
\usepackage{color}
\usepackage{listings}
\usepackage{caption}

% bibliography stuff
\usepackage{natbib}
\bibliographystyle{abbrvnat}
\setcitestyle{authoryear,open={(},close={)}}

% Pseudocode environment nicked from 
% https://tex.stackexchange.com/questions/111116/what-is-the-best-looking-pseudo-code-package


\lstnewenvironment{algorithm}[1][] %defines the algorithm listing environment
{   
    \captionsetup{labelsep=colon} %defines the caption setup for: it ises label format as the declared caption label above and makes label and caption text to be separated by a ':'
    \lstset{ %this is the stype
        mathescape=true,
        frame=tB,
        numbers=left, 
        numberstyle=\tiny,
        basicstyle=\scriptsize, 
        keywordstyle=\color{black}\bfseries\em,
        keywords={,input, output, return, datatype, function, in, if, else, foreach, while, begin, end, } %add the keywords you want, or load a language as Rubens explains in his comment above.
        numbers=left,
        xleftmargin=.04\textwidth,
        #1 % this is to add specific settings to an usage of this environment (for instnce, the caption and referable label)
    }
}
{}

\begin{document}
\maketitle
\begin{abstract}  % Does this go in the prelude or after the document begins?
\end{abstract}


\section{Introduction}


\section{Current Methods}
% Methods built on process structure trees


\subsection{SecSY}
% From the SecSY paper
SecSY\todo{CITE} is a tool produced with the intention of solving a lack of
``mechanisms for business process security monitoring and auditing''. In
particular, the work addresses the need for controlled log generation for
testing purposes with regards flexible and varying processes which exhibit
non-compliance. The authors' rationale is that, as the information systems
community increasingly develop tools which can test for certain traits in an
event log --- such as security policy violation --- their testing requires
suitable test data. A set of logs which exhibit the sought property, and a set
which do not, should be producable from a specification of ideal behaviour.
\par

The tool produces event logs exhibiting arbitrary properties which one might
want to observe in a controlled manner. It does this by executing a provided
process specification according to a ``context'', and recording the associated
event log (the simulation's ``trace''). The trace is then edited via a set of
transformers. The transformers used are well documented in \todo{CITE}, but
roughly correspond to switching gateway types, swapping order of task execution,
editing associated actors to introduce violation of authenatication, and editing
associated actors so that the same actor did not execute all of a set of
actions.
\par

The authors note that SecSY is ``capable of producing log files of industrial
complexity'', and that these logs are readily available to be used in testing by
the process mining community. The tool's evaluation affirms this by noting the
tool's runtime and that the transformers they detail are ``effective''. This
does not guarantee that transformers produce realistic data, however. We can
surmise that a transformer-based approach can be reliably and efficiently
implemented, but have no guarantees as to the \emph{realism} of the log it might
produce.
\par

We can expect that such a method would produce a realistic model only with
significant effort and changes. In the real world, logs produced with certain
pieces of variance ought to exhibit the impact of that variance on the decisions
made by agents in later stages of a process. Put another way, variance in the
real world produces state changes which have an effect on future decisions
actors might make. Post-processing on event logs would need to take this state
into account to predict the ``knock-on effects'' of behavioural variance, and
each sequential effect the initial change would make on state may have
cumulative impacts on the overall workflow execution.
\par

Therefore, while this method shows initial promise, amending it to satisfy the
requirement of realism in a generated event log seems intractable.
\par

\subsection{IBM paper}  % What's the name of this actual technique? TODO: Re-read the paper...

\subsection{Limitations of these Approach}
The limitation of this approach is evident when we consider what the impact of
contingent behaviour is on certain details of the world they are intended to
simulate. In the real world, the changes applied in ordinary process structure
tree models fail to represent how changes at one stage in a procedure impact
future behaviour.\par

A toy example of the real world might be a worker following a simple workflow on
a factory floor. The worker's workflow can be represented through a plethora of
standard modelling techniques, and the process fits the block-structured
workflow modelling required for analysis as a Process Structure Tree. A piece of
contingent behaviour which could be represented via these models might be the
actor executing the workflow being distracted.\par

Existing techniques, such as \todo{CITE}, represent these changes as an
amendment to the event log produced by a simulation from a Process Structure
Tree. \par




\section{Dynamic Fuzzing}


\section{Nuances and Definitions}



\section{BPMN representations in Python}




\end{document}
